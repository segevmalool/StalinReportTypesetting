\documentclass[a4paper, twocolumn]{article}
\usepackage[a4paper,
            includehead,
            headheight=0mm,
            footskip=7mm,
            left=20mm,
            right=20mm,
            bmargin=30mm,
            tmargin=10mm]{geometry}
\setlength\parindent{1cm}
\usepackage{lettrine}
\date{}
\renewcommand{\sffamily}{pte}

\title{Stalin's Great Terror Inspires the \emph{Short Course}:\\A New Perspective on
   the Soviet Communist Party}
\author{Michael Berton\\ Department of History \\ University of California, Santa Barbara}

%%%%%%%%%%%%%%%%%%%%%%%%%%%%%%%%%%%%%%%%%%%%%%%%%%%%%%%%%%%%%%%%%%%%%%%%%%%%%%%%%%%%

\begin{document}
\maketitle

\begingroup
\hyphenpenalty 10000
\exhyphenpenalty 10000


\lettrine[nindent=0cm]{O}{ften} considered the founder of communism,
Karl Marx coined the term ``superstructure''
to refer to the philosophical beliefs that legitimize the
credibility of an authoritarian figure or power, often a
government. The most direct attempt by the Soviet Communist Party to create a
superstructure, that is an indoctrinating, compelling story of its
rise to power, can be best symbolized by the \emph{Short Course of the Communist Party of the Soviet Union}. This book was orchestrated by Joseph Stalin, who ruled over the
world's first communist state, the Soviet Union, from 1924 until his
death in 1953. In addition to controlling a powerful, centrally
planned government, Stalin also purged anyone that he believed opposed
his regime during a period known as the reign of Great Terror. Stalin
justified the Great Terror by saying that those arrested were spies,
traitors, and enemies of Vladimir Lenin, Karl Marx, and the Communist
Party. Stalin also believed that people turned against the Party due
to their lack of knowledge about the history of the Communist Party in
the Soviet Union. Ultimately, Stalin wanted to correct for this
misguidance through the \emph{Short Course}. In short, realizing that the Soviet Union
lacked a historical narrative, Stalin desperately needed a way to
convince his own people, as well as the rest of the world, that his
regime was legitimate and noble. 
%2

Upon researching the various secondary sources available, I discovered
several competing historiographical interpretations of the \emph{Short
Course}. David Brandenberger, Professor at the University of Richmond
considered to be the most knowledgeable historian on this topic,
believes that the \emph{Short Course} symbolizes the larger movement of
Soviet patriotism.\footnote{Brandenberger, David, \emph{National Bolshevism}} While Brandenberger does emphasize the Great
Terror in his discussion of the historical background of the \emph{Short
Course}, he does not discuss the fact that the Great Terror actually
laid the seeds for the \emph{Short Course}. Another historian, Rusteem
Nureev, Professor of Economics at Moscow Higher School of Economics,
provided Stalin's speeches to the Politburo to convey Stalin's
rationale for the \emph{Short Course}.\footnote{Gregory, Paul. \emph{The Lost Politburo Transcripts}} Stalin realized that no dictator
could hold power if he was not supported by the nation's
intelligentsia. By eliminating anyone that opposed him during the
purges, Stalin had successfully erased the historical accounts of many
people that defied the Stalinist regime. In a way, the \emph{Short Course}
symbolizes the notion that the victors do indeed write history. This
point is proven by the fact that Stalin played such a huge role in
directing, writing, and editing the \emph{Short Course}. Stalin knew that
controlling the knowledge of his people was the best way to reinforce
his own power. Ultimately, Stalin dedicated so many hours to the
publication of the book because he believed that the Soviet Communist Party
had failed to produce a thorough story of its rise to power.
%3

This essay will not only elucidate the extent to which Stalin played a
vital role in the \emph{Short Course}, but it will also provide insights into
why Stalin felt that controlling the masses could best be done by
establishing an indoctrinating and compelling version of party
history. I will first discuss the countless revisions of the \emph{Short
Course} and then transition into Stalin's own editing of the
manuscript. The best sources for this essay come from the Russian
Digital Archive, sponsored by the Russian State Archive of Social
Political History (RGASPI) which include Stalin's edits of the second
edition as well as Stalin's letter to the writers of the second
edition of the \emph{Short Course}, Iaraslovaskii and Pespelov. With the help
of Professor Adrienne Edgar and Google Translate I was able to
translate portions of these texts. These sources illustrate
the notion that Stalin became monomaniacal with the book and
spent many months editing it. Furthermore, evidence from the \emph{Lost
Politburo Transcripts} will prove that Stalin himself saw the
\emph{Short Course} as an inevitable consequence of the Great Terror.

In essence, the \emph{Short Course} functions as the bible of the
Soviet Union. Stalin's strategy to reinforce his power was
twofold—first, to eliminate the opposition during the Great
Terror, and second, to rewrite history with the \emph{Short
  Course}. For this reason, the Great Terror and the \emph{Short
  Course} share a dependent historical relationship. Without the
Great Terror, members of the opposition would still be alive
to defy Stalin, and the book would have faced stronger
defiance and controversy. Without the book, Stalin would not
have been able to justify his own regime after the
controversial purges. One unusual aspect of the \emph{Short Course}
is that it was not designed for the working class but rather
the emerging Soviet intelligentsia. In doing so, Stalin
embraced a trickle down approach, realizing that to convince
the masses he first needed to indoctrinate the
educators. Ultimately, the \emph{Short Course} symbolizes Stalin's
attempt to legitimatize his own regime after the Great Terror
and provides valuable insights into how Stalin wanted to
portray himself to the rest of the world.


\begin{center}\textbf{Historical Background}\end{center}

In order to understand the relevance of the \emph{Short Course}, it is
essential to provide some historical background regarding
communism. In 1848, with their work The \emph{Communist Manifesto}, Karl Marx
and Friedrich Engels strongly advocated that communism was an
inevitable social-political advancement because industry-fueled
capitalism would fail to fairly distribute economic growth. Although
these intellectuals helped to spread communism, a book about party history in the
Soviet Union had failed to exist in the mid-late 1930s. Stalin saw the
failure to produce an accurate history as one of the greatest
downfalls of the Communist Party. Ultimately, the \emph{Short Course} served
as the most authoritative textbook on the history of the Communist
Party in the Soviet Union after Stalin's reign of Great Terror. This
was partly due to the fact that the Communist Party had recently
decided to disseminate the party's ideologies, considering the
millions of children that needed to be educated and indoctrinated into
Stalin’s communist system.\footnote{Brandenberger, David, \emph{National Bolshevism},p. 62} Additionally, Stalin was greatly
concerned with protecting his own image after the purges and did so by
promoting his own propaganda. As a result, historians often regard the
Soviet Union as the world's first true propaganda state. Stalin also
used his power to institute a series of Five Year Plans, massive
industrialization projects, and collective farming. Through his own
speeches to the Politburo in 1938, Stalin stated that he believed this
misguidance could have been prevented if the \emph{Short Course} had been
written earlier.\footnote{Gregory, Paul. \emph{The Lost Politburo Transcripts}, p. 165} In a way, this textbook symbolizes Stalin's
attempt to correct the critical mistake that caused the purges.

Historically, David Brandenberger discussed the deficiencies of the
educational system that swept the Soviet Union in the mid-1930s. In
addition, the widespread famine and purges of the 1930s led to a
strictly polarized political environment characterized by the
emergence of radical factions within the party. These officials often
had contrasting views about the proper interpretation of communism, and
could not reach an agreement about how to teach communism in
educational institutions. As a result, textbook publishers failed to
develop adequate textbooks for Soviet education.\footnote{ Brandenberger, David, \emph{National Bolshevism}, p. 63} This problem was
compounded by the fact that the emerging industrial society of the
Soviet Union in the 1930s resulted in population growth, which created
an even larger demand for educational resources. Consequently,
there existed an entire generation of students that needed guidance
about the history of the Communist Party. The official communist party
newspaper expressed this concern by publishing, ``Thirty million
school children need to be brought up in the spirit of boundless love
for the motherland and devotion to the party of Lenin and Stalin.''\footnote{ Brandenberger, David, \emph{National Bolshevism}, p. 64}
For all of these reasons, The \emph{Short Course} was a long-awaited history
textbook for many people. Thus, when the book was finally published,
it was widely embraced by intellectuals, educators, and the political
elite.


\begin{center}\textbf{The Origins of the First Edition}\end{center}

The origins of the first edition of the \emph{Short Course} were inspired by
the Politburo in 1936. On March 3rd, 1936, the Politburo decided to
launch a competition for the best narrative of political history in
the Soviet Union. The Politburo created a committee of judges to
assess the narratives and it was decided that the winning manuscript
would be made the Soviet Union's primary textbook on party
history. The Communist Party lacked a historical narrative and a central
doctrine.
  This competition directly laid the seeds for the \emph{Short Course} by
  asking intellectuals, historians, and public officials to write
  informative overviews of the Soviet Union. In response to its call-to-action, the Politburo received 42 manuscripts and selected seven
  to advance to the second stage of the competition.\footnote{Banerji, Arup. \emph{Writing History in the Soviet Union: Making the Past Work}. p. 66} The jury
  assigned to the competition chose to withhold the first and second
  place prizes to emphasize that the winner needed to drastically edit
  the version; they did not consider any of the manuscripts worthy of
  first place.  However, the jury did in fact award a third prize to
  A.S. Shestakov \emph{To the Happiest Children in the World}.\footnote{Banerji, Arup. \emph{Writing History in the Soviet Union: Making the Past Work}. p. 67} The
  competition then entered its third and final stage: the revision of
  the two best manuscripts.

The competition was interrupted because the timing between the second
and the third stage of the competition marked the height of the Great
Terror. This may have persuaded A.S. Shestakov to craftily revise his
original work in a way that supported Stalin. In spite of the Great
Terror, the competition resumed on August 22, 1937 and the revised
A.S. Shestakov's text, now called the original \emph{Short Course of the
Communist Party}, was awarded second place and won 75,000 ruples.\footnote{Banerji, Arup. \emph{Writing History in the Soviet Union: Making the Past Work. p.  68}}
The jury, once again, chose not to select a first place and instead
decided to publish Shestakov's version. Furthermore, A.S. Shestakov
co-wrote this book alongside N.G. Tarasov, N.D. Kuznetsov, and
A.S. Nifontov, members of the Moscow Pedagogical Institute. These
members were heavily involved in Soviet patriotism and propaganda and
had served the Communist Party since Lenin. Additionally, Andrei
Zhanov, the Party Manager of Leningrad, led these authors during the
revision process and personally wrote many sections of the book
himself.

After being revised for many months, the book was published in
September 1937.  The closing chapter of the introduction revealed the
overall purpose of the book.  The text read: ``We love our motherland,
and we must know her remarkable history well. Whoever knows history
will better understand current life, will fight the enemies of our
country better, and will consolidate socialism.''\footnote{Communist Party of the Soviet Union (1939). \emph{The Short Course VCP(b)}, p. 25} This could be
viewed as the maxim of the \emph{Short Course} and clarifies that its
function is to influence and to be shared. When the first edition was
released, publishing firms could not keep up with the high demand for
the \emph{Short Course} because the book was used by “political study circles
in factories and also by the armed forces”.\footnote{Gregory, Paul. \emph{The Lost Politburo Transcripts}, p. 69}For example, The Red
Army forced its soldiers to recite passages from the book, which
eventually turned into patriotic chants. Nonetheless, A.S. Shestakov
first edition was controversial and Stalin wanted to play an extremely
large role in editing the original version of the book, which, when
edited, was later referred to as the \emph{Short Course of the Communist
Party VKP(b)}.


\begin{center}\textbf{Stalin's Letter to the Writers of the Second Edition}\end{center}

	On April 6th, 1937, Stalin wrote a letter to Iaroslavskii and
    Pospelov in order to discuss some of the changes he wanted to see
    in their revised version. This letter was obtained from the Stalin
    Digital Archive and was translated by Google Translate. In the
    opening of the letter, Stalin revealed his frustrations with the
    current textbooks and stated,

\begin{quote}
``I think our textbooks on the history of the VKP(b) are unsatisfactory
for three principal reasons. They are unsatisfactory because they
present the history of the VKP(b) without a connection to the
country's history; or because they confine themselves to an account, a
simple description of the events and facts of the struggle among
currents without providing the necessary Marxist explanation; or
because they suffer from an incorrect construction and an incorrect
periodization of events.''\footnote{Stalin, Joseph (1937), \emph{Stalin's letter IV members of the Politburo of the CPSU (b), and the compilers of the textbook History of the CPSU (b), Stalin Digital Archive}}
\end{quote}

This quote shows that Stalin wanted the textbook to be grounded by
history and through Marxist ideology. In fact, Stalin's main proposal
in this letter was to add a paragraph to each section that provided
the reader with historical context as well as an explanation of
Marxist ideologies. To exemplify this, Comrade Zelenov translated a
segment of the letter and wrote,

\begin{quote}
``Stalin went on to write above the text on the proof of the article
(l. 34)." Each chapter (or section) of the textbook must be
prefaced with a historical background on the country's economic
and political situation. Without this, the history of the VKP(b) will
look not like history but like a lightweight account of past
events.''\footnote{ Stalin, Joseph (1937), \emph{Stalin's letter IV members of the Politburo of the CPSU (b), and the compilers of the textbook History of the CPSU (b), Stalin Digital Archive}}
\end{quote}


Indeed, Stalin wanted the \emph{Short Course} to appear as factual as
possible and did this by requiring that each section of the book
include a historical background. In this letter, Stalin also included
an outline to the book that completely reorganized the order of the
chapters:

\begin{quote}
``It is necessary, secondly, not only to present facts that demonstrate
the abundance of factions in the party and in the working
class during the period of capitalism in the USSR but also to provide
a Marxist explanation of these facts.''\footnote{ Stalin, Joseph (1937), \emph{Stalin's letter IV members of the Politburo of the CPSU (b), and the compilers of the textbook History of the CPSU (b)}, Stalin Digital Archive}

\end{quote}

This passage reveals how Stalin wanted to portray himself throughout
the book. In essence, Stalin wanted to show that his regime was
grounded by Marxist ideology. At the same time, Stalin knew that the
best way to support his regime was to argue that it was a rational
extension of Marxist ideology.


\begin{center}\textbf{The Second Edition}\end{center}

In April 1937, Stalin called for the second edition of the \emph{Short
Course} to be commissioned to leading historians Emil'ian
M. Iaroslavskii and Peter N. Pospelov. The original manuscript
exceeded 800 pages and Stalin found many errors within the original
text. Iaroslavskii, Pospelov, and a throng of historians from the
Marx-Engels-Lenin Institute spent many months revising the first copy
to Stalin's standards. Stalin was deeply involved in editing the text
and cut the original version by over 500 pages, inserting his own
sections into the book.

Ultimately, Stalin drastically changed the political narrative of the
Soviet Union by rewriting history itself. After the Great Terror, many
historians were executed and Stalin successfully intimidated members
of his own party, achieving total compliance. He was now in a position
to rewrite history.  Stalin met with the Central Committee in March
1937 to propose ``an ambitious new two-tiered party education system.''
The Politburo, almost entirely made up of sycophants, ratified Stalin's
proposals on April 16, 1937. These proposals called for ``another set
of curricular reforms'' and identified ``new indoctrination materials to
be published before that fall.''\footnote{Brandenberger, David. \emph{Stalin's Answer the National Question}} The \emph{Short Course} was
successfully able to finesse ``a delicate paradox: how could a
historical interpretation so much geared toward the valorization of
state authority explain the rise of antiestablishment revolutionary
movements.''\footnote{Brandenberger, David. \emph{National Bolshevism}, p. 53}

An analysis of certain passages from the second version of the \emph{Short
Course} depicts the party history as emerging from a proletariat
revolution. The first chapter discusses the abolition of serfdom and
the rise of industry in 19\textsuperscript{th} century Russia. In the author's view,
capitalism would inevitably produce an elite class and would cause
``the working class to develop its class consciousness, to organize it,
and to help it to create its own working party.''\footnote{Banerji, Arup. \emph{Writing History in the Soviet Union: Making the Past Work}, p. 67} The book goes on to
explain how the Bolshevik's October Revolution of 1917 was about
fighting for the rights of workers and peasants. One month after the
October Revolution, the newly formed Soviet Union entered into a Civil
War.  The Bolsheviks, represented by the Red Army, fought the White
Army in a series of battles that lasted over three years.

The Bolsheviks responded to the demands of war by adopting a policy
known as War Communism. This policy was highly controversial because
it made the government extremely powerful through the control of all
industry and agriculture. In a way, War Communism was a predecessor to
Stalin's massive industrialization plans and collective farming.
However, War Communism was detrimental to the peasants because it
required the seizure of portions of their crops. The \emph{Short Course}
states that this system ``introduced a state monopoly of the grain
trade and required peasants to be registered by the state''.\footnote{Bolsheviks, \emph{The Short Course of the CPSU ACP( b)}, p. 229}
Furthermore, the Bolsheviks had just seized power and it was essential
that they won the war. According to the text, ``the Bolsheviks
undertook intense preparations for a protracted war and
decided to place the whole country at the service of the
front.'' \footnote{Bolsheviks, \emph{The Short Course of the CPSU ACP( b)}, p. 229} War Communism was presented in a factual manner that
explained the necessities of the controversial policy. However, the
book does not address details like compensation for the workers and
farmers impacted by the policy, or the effect of the policy on their
standard of living.

Recognizing his audience, Stalin realized that the only way to win
over his people was to show a deep respect for Vladimir Lenin. Stalin
eulogized, ``Departing from us, Comrade Lenin adjured us to remain
faithful to the principles of the Communist International. We vow to
you, Comrade Lenin, that we will not spare our lives to strengthen and
extend the union of the toilers of the whole world.'' \footnote{Ibid, p. 269} However, the
book does not contain any information about how Stalin rose to power
and skips from discussions about Lenin's legacy to socialist
industrialization without discussing Stalin's rise to power. Stalin
did not see this as important, and instead choose to simply stress the fact
that he was a follower of Lenin. Furthermore, Stalin also wanted to
avoid any of the controversial stories that could arise from
explaining how he became the leader of the Soviet Union.

The final chapter ends with a discussion of the widespread acceptance
of the Bolshevik Party in the Supreme Soviet election of 1937. There
were 94 million voters and nearly 96.8 percent of them voted for the
Communist Party. This was considered ``a triumph for the Bolshevik
Party. It was a brilliant confirmation of the moral and political
unity of the Soviet people.''\footnote{Ibid, p. 352} This statistic is skewed by the fact
that the vote was conducted during the height of the Great Terror, and
Stalin had terrified the masses into aligning with the beliefs and
ideologies of the Soviet Communist
Party.


\begin{center}\textbf{Stalin's Revision of the Second Edition}\end{center}

	Stalin made countless edits in the book and played an extremely
    important role in the second edition. Evidence from Stalin's
    appointment books show that he had very few meetings scheduled in
    mid-May, mid-June, and mid-July, suggesting ``that it was during
    this time that he focused on the textbook''.\footnote{Brandenberger, David. \emph{Stalin's Answer the National Question}, p. 869} When revising the
    original text, Stalin ``struck entire paragraphs, pages, and
    subsections of the manuscript—-tens of thousands of words in
    all.''\footnote{Brandenberger, David. \emph{Stalin's Answer the National Question}, p. 873} Stalin often portrayed traitors within his own party as
    being ``well-organized domestic opposition with ties to enemies
    abroad''. Furthermore, Stalin's editing of the original account
    eliminated ``bourgeois nationalist conspiracies during the 
    Terror.''

As a result, Stalin's careful selection of Iaroslavskii and Pospelov's
original version emerged as a more direct form of propaganda. Stalin
managed to trim a fact-based document down to the portions
that supported his own regime. Stalin's revised version failed to
include any ``mention of non-Russians; political, economic, and
cultural modernization but also the entire subsection of Chapter 12
dedicated to the friendships of his peoples.''\footnote{Brandenberger, David. \emph{Stalin's Answer to the National Question}, p. 873}  Stalin also edited
the new version of the \emph{Short Course} and cut hundreds of pages from the
first edition.

Historiagraphically, The Yale University Press collaborated with the Russian State Archive
of Social and Political History (RGASPI) and translated certain
sections of Stalin's handwriting. Arguably, Stalin's propaganda,
ideology, and strategy to reinforce his power were successful because
he ruled the Soviet Union for over 25 years. The last portion of this
section will provide some of the direct writing from Stalin as he
edited the second edition. For example, Stalin decided to delete the
text below from the second edition:

\begin{quote}
``Already in this period of these enemies of Bolshevism, despite the ``nuances'' of their views, disbelief in the victory of the socialist proletariat revolution is born.
The Party, under the leadership of Lenin and Stalin, had successfully
organized workers and toiling peasants for the overthrow of the
bourgeoisie and the landlords.''\footnote{Stalin, Joseph. \emph{The Original Text of the Short Course}, Stalin Digital Archive: RGASPI and the Yale University Press.  F. 558 op. 11. d.1211}
\end{quote}

Stalin had recently executed millions of his opponents, successfully
erasing accounts of his opposition from history. By launching the
purges, Stalin put himself in the unique position of being able to
rewrite history to his own liking. Consequently, the \emph{Short Course
VKB (b)} would eventually become the sole textbook for all grades,
ranging from elementary to graduate schools.

Foreign affairs also influenced the revisions of the text,
particularly Germany's rise to prominence in the late 1930's. By the
end of the decade, Hitler had transformed Germany into one of the most
industrialized, militarized superpowers in the world. Stalin was
fearful that the Germans would spread their ideology to the Soviet
Union in an attempt to destroy the Soviet Communist Party. Toward the end of
the \emph{Short Course}, Stalin inserted the following passage:

\begin{quote}
``The victory of fascism in Germany must be regarded not only as a sign
of weakness of the working class and the result of changes of social
democracy to the working class, which paved the way for fascism. It
must also be regarded as a sign of weakness of the bourgeoisie, as a
sign that the bourgeoisie is already unable to rule by the old methods
of parliamentary and bourgeois democracy, which is why it is forced
to resort to domestic policy to terrorist methods of management
...'' \footnote{Stalin, Joseph. \emph{The Original Text of the Short Course}, Stalin Digital Archive: RGASPI and the Yale University Press.  F. 558 op. 11. d.1211}
\end{quote}

This passage also illustrates Stalin's advocacy of the
proletariat. Stalin wanted his people to believe that he truly valued
the working class, and was in accordance with Marx's vision that the
proletariat should overthrow the bourgeoisie. Additionally, Stalin
also wanted to show his own people that his political party and his
regime were grounded by the tenets of true Marxism.

	To this end, Stalin discussed at length the reasons why the kulak
    class was destructive and detrimental to the Soviet people. For
    example, Stalin wrote, ``It eliminated the most numerous class of
    exploiters in our country, the class of the kulaks... It furnished
    the Soviet regime with a Socialist base in the most extensive in
    the most necessary, but also in the most backward areas of the
    economy- agriculture.''\footnote{Stalin, Joseph. \emph{The Original Text of the Short Course}, Stalin Digital Archive: RGASPI and the Yale University Press.  F. 558 op. 11. d.1211} The insights gained from these revisions
    show how Stalin wanted to rewrite history to his own
    agenda. Stalin felt so strongly about the \emph{Short Course} that he
    spent months editing it to his liking. In short, the \emph{Short Course}
    best symbolizes the larger movement of nationalism inspired by
    Soviet propaganda and indoctrination.

It is important to emphasize the fact that this archival source is
very incomplete because it provides less than twenty percent of
Stalin's original edits. As a result, it is difficult to draw
conclusions from the quotes found above. Furthermore, only certain
portions of the archive had been translated. Stalin's handwriting is
notoriously bad, and it has been difficult for translators to
understand many aspects of the text. These are some of the limitations
of this source. However, the source itself does illustrate the
incredible amount of time Stalin spent revising the course, and
sparked new questions for my research. For example, why was Stalin
interested in this particular book when he had an entire nation to
lead? When in modern day history have we seen a leader spend so much
time on a history textbook?


\begin{center}\textbf{Stalin's Speeches to the Politburo}\end{center}

Stalin wanted the book to be directed toward the emerging Soviet
intelligentsia. The Politburo, regarded as the main authority of
Soviet ideology and propaganda, held a session to address the
publication of the \emph{Short Course} from October 22\textsuperscript{nd} to October 23\textsuperscript{rd},
1938. The session was called, ``On the Question of Party Propaganda in
the Press Associated with the Publication of the \emph{Short Course},” and
was presented by Andrei Zhdanov, now Stalin's main advisor on
propaganda.''\footnote{Gregory, Paul, \emph{The Lost Politburo Transcripts}. p.166} Stalin himself played an essential role in the
meeting. The \emph{Lost Politburo Transcripts} provides Stalin's speeches,
which are particularly revealing of his intentions with the \emph{Short
Course}. He discussed the function of the book and whom he wanted to
most strongly influence. During the opening discussion, Andrei Zhdanov
stressed the fact that the \emph{Short Course} was ``targeted primarily at our
leading officials, at our Soviet intelligentsia.'' Similarly, Stalin
stated,

\begin{quote}
``For whom is this book? It is for the cadres, for our cadres. And what
are cadres? They are the command staff, the lower, middle, and higher
command staff of the entire status apparatus. From now on our
propaganda should address our intellectual cadres.''\footnote{Ibid, p. 169}

\end{quote}

Soviet education was rapidly improving during the 1930's and a larger
class of intellectuals began to sweep across the Soviet Union. Thus,
Stalin realized that the support of the intelligentsia was integral
for the continuation of his regime. Similarly, Stalin addressed the
Politburo, ``There is no class that can maintain its domination and
rule the state if it is not capable of creating its own
intelligentsia.''\footnote{Ibid, p. 168. p.166} Stalin believed that the intellectuals strongly
influenced the masses by serving as either public officials or by
organizing the proletariat into political groups. Most importantly,
these intellectuals had tremendous power over the worker's
understanding of Stalin's regime.  However, Stalin also believed that
the intellectuals were plagued by a lack of knowledge about the
history of the Communist Party. He described this problem to the
Politburo when he stated,

\begin{quote}
``An official is an individual who makes conscious decisions. He wants
to know what is going on, he raises questions, gets confused because
he does not have adequate understanding of politics, preoccupies
himself with petty trifles, exhausts himself; finally he loses
interest in Marxism and in his bolshevization. We ought to compensate
for this failure of ours…and the best way to begin is to publish the
\emph{Short Course}.''\footnote{Gregory, Paul. \emph{The Lost Politburo Transcripts}. p. 171}
\end{quote}


This quotes proves that Stalin's main purpose in writing the \emph{Short
Course} was to provide a detailed history about the Soviet Communist Party in
order to persuade intellectuals and public officials to support his
regime.  Furthermore, the motivation for propaganda that addressed the
intellectual elite was partly grounded by Stalin's purges, which
lasted from 1936 to 1938. As a result, Stalin appointed authorities
who not only agreed with everything he said, but who were also terrified
to disagree. For example, Comrade Andropov from the Orel Party
Committee, who headed the propaganda department, reported to Stalin,
``If you take the case of editors, then we have 99 percent new
cadres.''\footnote{Gregory, Paul. \emph{The Lost Politburo Transcripts}. p. 166} Stalin addressed the need for these purges at the
Politburo meeting and declared,

\begin{quote}
``The most serious evil, which we uncovered in the recent past, was
that our cadres were not satisfactorily equipped. These were cadres
that could not digest the sharp turn to collective farms, they could
not envision such a change because they were not politically
equipped.''\footnote{Gregory, Paul. \emph{The Lost Politburo Transcripts}. p. 166}
\end{quote}

Stalin rationalized that the purpose of the purges was to dispel the
Communist Party of members that were not ``politically equipped.'' These
were cadres that had a lack of understanding about collective farms
and rapid industrialization and lost sight of Stalin’s goals.  Stalin
believed that these members were misguided because they did not have a
central authoritative book on the history of the last 20
years. Previously, the  main book that intellectuals read that
reinforced the tenets of communism was Karl Marx and Friedrich Engel’s
\emph{Communist Manifesto}. The \emph{Communist Manifesto} was powerful because it
convinced intellectuals and the proletariat to pursue
communism. However, while this book explained the central belief that
communism was an inevitable step in political development in a highly
industrialized world, and that capitalism would fail to distribute
economic growth among the masses, it did not provide an authoritative
account of the history of the first communist state. In response to
these circumstances, Stalin stated,

\begin{quote}
``They did not know the laws of societal development, of economic
development, of political development. How can we explain that some of
them became spies and intelligence agents? Some were our own people,
who went over to them. Why? It appears that they were not politically
equipped.''\footnote{Gregory, Paul. \emph{The Lost Politburo Transcripts}. p. 170}
\end{quote}

This shows that Stalin believed that people who opposed him were
widely dispersed in his own party and that they even served as spies
and traitors. Stalin also discussed that the purges created some
positive opportunities for his new vision of the Soviet Union. Stalin
expressed these positive opportunities and also believed that
if the \emph{Short Course} had been written earlier, the Soviet
Communist Party could have prevented the purges,

 \begin{quote}
``At this time, we lost a part of our cadres, but we gained an enormous
number of lower-level workers, we got new cadres, we won over the
people to collective farms, we won over the peasantry. They must be
directed through the administration, but in this administration, there
were it appears, the wrong people. In this fashion, in winning the
people we let an opportunity slip by our cadres. We must recognize
this fact, and we must correct this mistake.''\footnote{Gregory, Paul. \emph{The Lost Politburo Transcripts}. p. 170}
\end{quote}


The phrase "they must be directed through administration" illustrated
Stalin's strong desire for the creation of the \emph{Short Course}. By
failing to win the support of the Soviet people with a political
narrative, Stalin realized that he had people in the administration
that were seen as the "wrong people", and hoped that the creation of
the \emph{Short Course} would "correct this mistake." This verifies the
historical interdependence of the \emph{Short Course} and the Great
Terror. In essence, Stalin wanted to use the \emph{Short Course} as a means
to win over the people and to prevent future purges.

Rustem Nureev, Professor of Economics at the Higher School of
Economics in Moscow, supported this notion when he wrote, ``the
implication of Stalin's indirect remarks on the Great Terror is that
if there had been a \emph{Short Course} earlier, cadres would have understood
his policies better, would have supported them, and mass purges would
not have been necessary.''\footnote{Gregory, Paul. \emph{The Lost Politburo Transcripts}. p. 170} Stalin confirmed this when he told the
Politburo,

\begin{quote}
``Correcting this mistake begins with the publication of the \emph{Short
Course}. This book demonstrates the basic ideas of Marxism-Leninism on
the basis of historical facts. Because it demonstrates its theses with
historical facts, it will be convincing for our cadres, who work with
their intellects, for thinking people who will not blindly follow. We
have not paid sufficient attention to this matter and now we must
complete it''.\footnote{Ibid. p. 170}
\end{quote}


Stalin also realized that the creation of the \emph{Short Course} was only
going to be successful if it was widely embraced by
intellectuals. To this end, Stalin summoned hundreds of party
propaganda workers to his meeting with the Politburo, and dedicated
part of the meeting to addressing the efficiency of the workers. For example, Rustem Nureev wrote, ``Were those party officials responsible
for the dissemination of ideology up to the task? The party leadership
realized the discrepancy between the rising cultural level of the
workers and the intelligentsia and the reasons to doubt their
effectiveness.'' \footnote{Gregory, Paul. \emph{The Lost Politburo Transcripts}. p. 171} Soviet officials assessed the effectiveness of
propaganda workers by the number of attendees at their meeting. The
Ivanovo Party Committee told Stalin that the number of people at their mandatory
reading of the \emph{Short Course} had risen from 48,000 to 104,000 in only three months.\footnote{Ibid. p. 174}

Additionally, Rustem Nureev analyzed Stalin's transcript to the
Politburo and stated, ``Stalin's message called in the declining days
of the Great Terror, was that he was ready to turn from physical
elimination of enemies to enemy prevention.''\footnote{Ibid. p. 174} The purges were over
and Stalin had successfully wiped the slate clean for the Soviet Communist
Party to rewrite history. Now that all the disbelievers were dispelled, Stalin was able
to start over and indoctrinate intellectuals with his own version of
party history. The \emph{Short Course} was, in a way, an inevitable
consequence of the purges. Stalin realized that the ``mistake'' that led
to the purges was the failure to promote the full history of the
Soviet Communist Party. Now that the party was cleansed, Stalin was in an
ideal situation, not only to promote his own understanding of Soviet
history, but also to rewrite history itself.


\begin{center}\textbf{The Great Terror in the \emph{Short Course}}\end{center}

	The \emph{Short Course} emphasized that the Great Terror was justified
    because the traitors opposed Lenin and the Soviet Communist Party since
    the October Revolution of 1917. The book does not mention anything about
    the Gulags and only discussed the Party leaders who were tried and
    executed. Since the book is chronological and ends in 1937, the
    \emph{Short Course} only touches on the Great Terror, in four pages at
    the end of the book. The purges are addressed in Chapter 12,
    titled, ``Liquidation of the Remnants of the Bukharin-Trotsky Gang
    of Spies, Wreckers and Traitors to the Country.'' Stalin justified
    these executions by saying that these traitors had been in
    conspiracy with Lenin since the October Revolution in 1917. The
    text stated, ``The trials showed that these dregs of humanity, in
    conjunction with the enemies of the people, Trotsky, Zinoniev and
    Kamenev, had been in conspiracy against Lenin, the Party and the
    Soviet state ever since the early days of the October Socialist
    Revolution.''\footnote{Bolsheviks, \emph{The Short Course of the CPSU ACP(b)}, p. 347} The writers of the \emph{Short Course} and Stalin wanted
    to paint the opposition in defiance of Lenin and Marx. For
    example, the book stated,

\begin{quote}
``The trials brought to light the fact that Trotsky-Bukharin fiends, in
obedience to the wishes of their masters, the espionage services of the
foreign states—had set out to destroy the Party and the Soviet state,
to undermine the defensive power of the country, to assist foreign
military intervention, to prepare the way for the defeat of the Red
Army, to bring about the dismemberment of the U.S.S.R., to hand over
the Soviet Maritime Region to the Japanese, and to destroy the gains
of the workers and collective farmers, and to restore capitalist
slavery to the U.S.S.R.''\footnote{Bolsheviks, \emph{The Short Course of the CPSU ACP( b)}, p. 347}
\end{quote}


This excerpt shows that the book strategically answered the question of
the Great Terror by emphasizing that the opposition was directly
involved in treason and also wanted to undermine the Communist Party
and Stalin's rule. By showing that the purges were justified by
treason, the book is able to argue that the executions were necessary.
This was because believers in communism regarded Marx and Lenin as their true heroes.


\begin{center}\textbf{Impact of The \emph{Short Course} on the CPSU}\end{center}

The fact that the textbook was designed for the Soviet
intelligentsia made it somewhat inaccessible to the
masses. Although Bukharin had come out with the \emph{ABCs of Communism},
a book designed for the masses, Stalin did not regard this as
worthy  because he wanted a book that was tailored to the
intelligentsia. Stalin realized that he could never hold onto his
power unless he had the support of the intellectuals, public
officials and educators. Since many people in the Soviet Union did
not have more than an elementary education, it was understood that
the common people would need direction and guidance with the \emph{Short
  Course}. These educators would ultimately be the ones responsible
for interpreting the book and teaching it to the masses. As a
result, Stalin realized that winning over the intellectuals
ultimately meant winning over the working class.  Furthermore, it would be
difficult to win over the mindsets of the intellectuals if there
were competing versions of Soviet history, a problem that Stalin had
rectified with the Great Terror.

The book's appeal to the masses was indirect; the intellectuals would spread the knowledge to the
masses.  As a result, many historians would argue that the \emph{Short
Course} might have been more successful at promoting Soviet ideology if
it were written for the masses instead of the intellectuals. However,
Stalin was desperate to maintain his power, and this meant appealing
to the logic of intellectuals and public officials. The common people
were powerless in the Soviet Union and public and government officials
maintained tight control over their people. Since the Soviet Communist Party was a
central power with a versatile range of power, Stalin must have
realized that he most needed the support of the intellectuals and
officials who could influence the masses to support the Stalinist
regime. To shed light on the hegemony of the \emph{Short Course},
K.F. Shteppa, professor at Kiev University during the 1930s, recalled,

\begin{quote}
``It was the only material on Russian history for courses in these and
even in the collegiate-level schools. Only by means of this little
book was it possible to orient oneself regarding the demands of Party
policy with respect to any historical question, phenomenon, or
event.''\footnote{Brandenberger, David, \emph{National Bolshevism}, p. 55}
\end{quote}


These are the main reasons why the \emph{Short Course} failed to garner the
support of the masses. In response to these concerns, Stalin met with
the Politburo in October 1938 to propose a revised edition of the
\emph{Short Course}. However, the new \emph{Short Course} failed to include history after 1938. As a result, the book became obsolete
by the mid-1940s and failed to discuss anything about World War
II. Additionally, World War II induced a different political and
economic agenda for the Soviet Union. Intellectuals and officials felt
disconnected with the goals of the Soviet Union due to the different
political and economic climates created by the war. Overall, the \emph{Short
Course} was not only difficult for the masses to read, but it also
failed to provide an up-to-date record of Soviet history.


\begin{center}\textbf{Conclusion}\end{center}

The \emph{Short Course} represented the second phase of Stalin's
indoctrination strategy, the first phase being the purges. Therefore,
the purges and the book are inextricably tied to one another; they
were both part of Stalin's overarching plan to rewrite history. The \emph{Short Course} was convincing because it was presented with
factual information and encompassed many aspects of Soviet life. The
full 350 pages presented its material as impartial and
factual. Additionally, the original book covers many aspects of
political history ranging from the Bolshevik Revolution to the
worker's opposition. In short, Stalin not only controlled the lives of
his own people, but their mindsets as well.  The fact that the book
was directed towards the Soviet intelligentsia suggests the fact that
Stalin was primarily concerned with indoctrinating educators and
historians.

Evidence from the \emph{Lost Politburo Transcripts} proved that Stalin felt
that the purges could have been prevented if the \emph{Short Course} had been
written earlier. Stalin himself cited that the two have a close
historical relationship. Although Stalin believed that his desire for the
book emerged after the purges, I believe that the purges inspired the
book. After the Great Terror, Stalin put himself in a position in
which he could rewrite history. While the book could prevent future purges
from occuring, I do not think that the \emph{Short
Course} could have prevented the Great Terror. Stalin was only
interested in writing the \emph{Short Course} after the Great Terror because
he knew that he had successfully eliminated all historical accounts
that could undermine the \emph{Short Course}. Thus, the book could be widely
spread without too much controversy. After wiping the slate clean,
Stalin could promote whatever ideology he wanted without much
defiance.

Overall, the \emph{Short Course} offers valuable insights into how Stalin
wanted to portray himself to his people and to the rest of the
world. Some of the most powerful themes that emerge from the book are
the values of the proletariat class, the worker's revolution,
Leninism, and industry. By 1940, more than four million copies of the
book had been published, and had become widely adopted in thousands of
reading circles throughout the Soviet Union. For all of these reasons,
the \emph{Short Course} is indeed the bible of the Soviet Union. In essence, the \emph{Short Course} could symbolize what
Marx would refer to as the ``superstructure'' of the Stalinist regime,
making it an extremely valuable historical artifact from the Soviet
Union.


\clearpage

\endgroup
\onecolumn
\renewcommand*{\thefootnote}{\fnsymbol{footnote}}

\begin{thebibliography}{10}
\setlength{\itemindent}{-0.3in}

\bibitem{}  Banerji, Arup (2008). \emph{Writing History in the Soviet Union: Making the
Past Work}. New Delhi, India: Social Science Press.

\bibitem{}  Brandenberger, David (2002). \emph{National Bolshevism}. Boston,
Massachusetts: Harvard University Press.

\bibitem{}  Brandenberger, David \& Zelenov, Mikhail. Stalin’s Answer to the
National Question: A Case Study on the Editing of the 1938 \emph{Short
  Course}. New Haven, Connecticut: Yale University Press.

\bibitem{}  Communist Party of the Soviet Union (1939). \emph{History of the Communist
Party of the Soviet Union ACP (b)}. Moscow, Russia: Foreign Languages
Publishing House.

\bibitem{}  Gregory, Paul, \& Maimark, Norman (2008). \emph{The Lost Politburo
Transcripts}, New Haven, Connecticut: Yale University Press.

\bibitem{}  Stalin, Joseph (1938). The Original Text of the \emph{Short Course} of
the Communist Party ACP (b). New Haven, Connecticut: Yale University
Press \& Moscow, Russia: Russian State Archive of Social Political
History, retrieved May 21, 2015 from the Russian Digital Archive.


\bibitem{}  Stalin, Joseph (1937) \emph{Stalin's letter IVmembers of the Politburo of
the CPSU (b)}, and the compilers of the textbook History of the CPSU
(b) New Haven, Connecticut: Yale University Press \& Moscow, Russia:
Russian State Archive of Social Political History, retrieved June 2,
2015 from the Russian Digital Archive.\footnote[1]{
The last two references come from the Stalin Digital Archive, a
research collaboration between the Russian State Archival of
Social-Political History and the Yale University Press. With the help
of Ann Platoff at the UCSB library, the University was able to obtain
a one-month free trial to the archive. Additionally, Google helped me
tremendously by translating the text while Professor Edgar of the UCSB
History Department personally helped me with several of the
translations. Special thanks to everyone involved that helped me
obtain access and understand this source.
}
\end{thebibliography}
\end{document}























